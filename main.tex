\documentclass[handout]{beamer}

% Choose how your presentation looks.
%
% For more themes, color themes and font themes, see:
% http://deic.uab.es/~iblanes/beamer_gallery/index_by_theme.html
%
\mode<presentation>
{
  \usetheme{CambridgeUS} %JuanLesPins}      % or try Darmstadt, Madrid, Warsaw, ...
  \usecolortheme{beaver} % or try albatross, beaver, crane, ...
  \usefonttheme{serif}  % or try serif, structurebold, ...
  \setbeamertemplate{navigation symbols}{}
  \setbeamertemplate{caption}[numbered]
} 

\usepackage[english]{babel}
\usepackage[utf8x]{inputenc}
%\usepackage{wrapfig}
%\usepackage{graphicx}
%\usepackage{hyperref}
\usepackage{multirow}
\usepackage{listings}


\title[DAGSI Update - RY9-UC-20-4]{Adaptable Protection for Embedded Systems Resilience}
\subtitle{ARFL Presentation - Fall 2020}
\author[B. King \& R. Jha]
       {\parbox[t]{1.5in}{Bayley King\\Graduate Student\\\texttt{king2b3@mail.uc.edu}} \and 
        \parbox[t]{1.5in}{Rashmi Jha\\Advisor\\ \texttt{Rashmi.jha@uc.edu}}}
%\institute{MIND Lab \newline
%University Of Cincinnati}
\institute[UC]{MIND Lab \and University of Cincinnati}

\date{September 4, 2020}
\logo{\includegraphics[height=1.5cm]{Images/ucLogo.png}}

\begin{document}

\begin{frame}
  \titlepage
\end{frame}

% Uncomment these lines for an automatically generated outline.
\begin{frame}{Outline}
  \tableofcontents
\end{frame}

% Define what happens after each section after the Introduction section.
\iffalse
\AtBeginSection[] {
  \begin{frame}<beamer>
  \frametitle{Outline}
  \tableofcontents[currentsection, currentsubsection]
  \end{frame}
}
\fi


\newcommand{\lenitem}[2][.85\linewidth]{\parbox[t]{#1}{\strut #2\strut}}
\begin{frame}{Differences Between GA and HereBoy for Genetic Programming}
\begin{itemize}
  \item Population Function
  \begin{itemize}
    \item GA initializes a population of randomly generated ASTs
    \item HereBOY uses only a single ASTs
  \end{itemize}
  \item Fitness Function
  \begin{itemize}
    \item Each circuit is scored off of their logical functionality 
    \item Each node in the AST is scored individually, by seeing which modification to that node would give the highest overall fitness
  \end{itemize}
  \item Selection Function
  \begin{itemize}
    \item The best preforming individuals of the population are selected for re-population
    \item The modification that offers the highest overall fitness is selected
  \end{itemize}
  \item Mutation Function
  \begin{itemize}
    \item \lenitem{The highest preforming individuals are then mutated to generate a new population}
    \item \lenitem{The new AST is formed by implementing the modification selected in the past stage}
  \end{itemize}
\end{itemize}
\end{frame}

\section{Work Completed}


% single figure slide
%\begin{frame} 
%  \begin{figure}
%    \includegraphics[width=\textwidth]{Images/uml.png}
%  \end{figure}
%\end{frame}



\iffalse  % multiline comment
% Example of using listing for code 
\defverbatim[colored]\lstI{
\begin{lstlisting}[language=Verilog,basicstyle=\ttfamily,keywordstyle=\color{red}]
  module  main(X, Y, Z, O);
  input X, Y, Z ;
  output O;
  wire  O;
  assign O =~ (X & Y) | (Y & Z) | (X & Z);
  
  endmodule

  [<HdlBuiltinFn.OR: 24>(<HdlBuiltinFn.OR: 
  24>(<HdlBuiltinFn.NEG: 20>(<HdlBuiltinFn.AND: 
  23>('X', 'Y')), <HdlBuiltinFn.AND: 23>('Y', 'Z')), 
  <HdlBuiltinFn.AND: 23>('X', 'Z'))]
\end{lstlisting}
}

\begin{frame}{Verilog to AST}
\lstI
\end{frame}
% end of example of listing for code

% example of listing for normal usage
\defverbatim[colored]\lstI{
\begin{lstlisting}[language=C++,basicstyle=\ttfamily,keywordstyle=\color{red}]
  s a v e -  -  -	   s a - v -  - e
  s a l v a  g  e	   s a l v a  g e 
      5 steps		       3 steps
  naive comparison       Levenshtein distance
\end{lstlisting}
}

\begin{frame}{Levenshtein Distance}
\lstI
\end{frame}

\fi

\section{Conclusions}
\begin{frame}{Current Issues}
  \begin{itemize}
    \item Fix structural fitness
    \item Mutations currently only work for 2-terminal logical feature (AND OR)
    \item Implement other logical gates (MIN, MAX, XOR, etc)
    \begin{itemize}
      \item hdlConvertor seems to have 27 different functions
    \end{itemize}
    \item Get each function working together properly
    \item hdlConvertor seems to be going through a fundamental rework, I'm having to manually input each AST to run tests
  \end{itemize}
\end{frame}


\begin{frame}{Plan for the Rest of the Semester}
  \begin{itemize}
    \item Start testing out scalability  larger circuits
    \begin{itemize}
      \item Will the system need to be modified to work for larger circuits?
    \end{itemize}
    \item Get code onto FPGA
    \item Looking for HOST 2021 submission deadline in March 21
  \end{itemize}
\end{frame}
  
  \begin{frame}{References}
    \begin{itemize}
      \item [1] Zhang K. and Shasha D., “Simple fast algorithms for the editing
      distance between trees and related problems” SIMA J. Comput.,
      18(6):1245– 1262, 1989 \newline
      \item [2]Xu H., “An Algorithm for Comparing Similarity Between Two Trees”. ArXiv:1508.03381v1, 2014
    \end{itemize}
  \end{frame}


\end{document}
